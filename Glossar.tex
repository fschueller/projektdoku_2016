%!TEX root = Projektdokumentation.tex
% FIXME toms: Glossar sortieren
% FIXME toms: evlt. das Glossar über \begin{glossary} ... \end{glossary}
%             erzeugen.
% FIXME toms: Sollten GitHub, Trello, CSS aufgenommen werden?

\subsection*{Self-Service-Technologies}
Technische Kommunikationsmöglichkeiten, die Nutzer zur selbstständigen Inanspruchnahme einer
Dienstleistung befähigen. Sie werden hauptsächlich zur Entlastung des Personals von
Standardaufgaben eingesetzt.
\subsection*{eDirectory}
% FIXME toms: Evlt. Active Directory ins Glossar mitaufnehmen? So
% definierst du ein Verzeichnisdienst über einen anderen der nicht im
% Glossar enthalten ist
Von Novell entwickelter Verzeichnisdienst, welcher ähnliche Funktionen wie das Active Directory
aufweist.
\subsection*{Test Driven Development}
% FIXME toms: "agile Entwicklungsmethode" hmn... würde ich anders schreiben:
% \emph{Testgesteuerte Programmierung} ist ein
% Software-Entwicklungsprozess. Es basiert auf dem Prinzip, dass ...
Dt. testgesteuerte Programmierung, ist eine agile Entwicklungsmethode. Bei dieser Methode werden
Software-Tests vor den zu testenden Komponenten entwickelt.
\subsection*{Ruby}
Höhere Programmiersprache, welche im Jahr 1990 von Yukihiro Matsumoto entwickelt wurde. Ruby
ist vollständig objektorientiert und wird zur Laufzeit interpretiert. Sie folgt dem sogenannten
\anf{Prinzip der geringsten Überraschung}, welches besagt, dass Programmierer eine Sprache möglichst
intuitiv und ohne Verhaltensanomalien der Sprache nutzen können.
\subsection*{Ruby on Rails}
Das Web Application Framework von Ruby. Es ist von den Prinzipien \anf{don't repeat yourself} (DRY) und
\anf{convention before configuration} geprägt, welche besagen, dass Code möglichst ohne Duplikation
geschrieben und die Konventionen bei der Namensgebung von Objekten eingehalten werden sollen.
\subsection*{Funktionale Programmierung}
% FIXME toms: evlt. Beispiele von typischen funktionalen Programmiersprachen?
Programmierparadigma, bei welchem Programme ausschließlich aus Funktionen bestehen.
\subsection*{Imperative Programmierung}
Programmierparadigma, bei welchem Programme aus einer Folge von Anweisungen bestehen.
\subsection*{Objekt (objektorientierte Programmierung)}
Stellt ein Exemplar eines Datentyps oder einer Klasse dar. Sie werden zur Laufzeit erzeugt und haben
einen Zustand (Attribute), ein Verhalten (Methoden) und eine Identität (Einzigartigkeit des Objekts trotz
Existenz gleichartiger Objekte)
\subsection*{Gem}
Auch RubyGems genannt, stellt das Paketsystem der Programmiersprache Ruby dar. Mit der Nachinstallation
von Gems kann Ruby um weitere Funktionen erweitert werden.
\subsection*{Flag}
In der Informatik als Statusindikator eingesetzt. Häufig wird dafür eine boolesche Variable genutzt,
welche die Zustände \anf{wahr} oder \anf{falsch} annehmen kann.
\subsection*{Twitter Bootstraip}
% FIXME toms: Web-Framework? CSS-Framework?
Ein von Twitter entwickeltes Framework, welches Gestaltungsvorlagen für
Typografie, Formulare, Buttons, Tabellen, Grid-Systeme, Navigations- und andere
Oberflächengestaltungselemente umfasst und zusätzliche Erweiterungen für JavaScript enthält.
\subsection*{Single Table Inheritance}
Dt. Tabelle pro Vererbungshierarchie, verwendet eine einzige Tabelle für jede Klasse, um einen
Klassenbaum in einer Datenbank abzubilden.
\subsection*{Partial View}
Dynamische Komponenten eines sonst statischen HTML/CSS-Views.
\subsection*{Active Record}
Konzept für objektorientierte Software zur Speicherung von Daten in einer relationalen
Datenbank.
\subsection*{Active Job}
Klassentyp in Ruby on Rails, welcher zur Ausführung von Hintergrundprozessen, beispielsweise
externe API-Anfragen, existiert.
\subsection*{Feature Test}
Testmethode zum Testen einer kompletten Funktionalität (Feature) und dem Zusammenwirken der
Softwarekomponenten im Hintergrund.
\subsection*{Usability Test}
Testmethode zum Testen der Gebrauchstauglichkeit einer Software mit den potentiellen Benutzern.
\subsection*{RPM Package Manager}
Freies Paketverwaltungssystem, ursprünglich entwickelt von Red Hat.
