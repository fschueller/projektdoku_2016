% !TEX root = ../Projektdokumentation.tex
\section{Projektplanung}
\label{sec:Projektplanung}


\subsection{Projektphasen}
\label{sec:Projektphasen}

% \begin{itemize}
% 	\item In welchem Zeitraum und unter welchen Rahmenbedingungen (\zB Tagesarbeitszeit) findet das Projekt statt?
% 	\item Verfeinerung der Zeitplanung, die bereits im Projektantrag vorgestellt wurde.
% \end{itemize}
%
% \paragraph{Beispiel}
% Tabelle~\ref{tab:Zeitplanung} zeigt ein Beispiel für eine grobe Zeitplanung.
% \tabelle{Zeitplanung}{tab:Zeitplanung}{ZeitplanungKurz}\\
% Eine detailliertere Zeitplanung findet sich im \Anhang{app:Zeitplanung}.
%
%
% \subsection{Abweichungen vom Projektantrag}
% \label{sec:AbweichungenProjektantrag}
%
% \begin{itemize}
% 	\item Sollte es Abweichungen zum Projektantrag geben (\zB Zeitplanung, Inhalt des Projekts, neue Anforderungen), müssen diese explizit aufgeführt und begründet werden.
% \end{itemize}

\subsection{Ressourcenplanung}
\label{sec:Ressourcenplanung}
\subsubsection{Personalplanung}
\label{sec:Personalplanung}
Das für die Entwicklung benötigte Personal besteht aus einem Auszubildenden, der dafür von seiner
regulären Mitarbeit im derzeitigen Team freigestellt wird.
\subsubsection{Kostenplanung}
\label{sec:Kostenplanung}
Die Personalkosten belaufen sich auf ein Auszubildendengehalt, wodurch lediglich Stromkosten für die
zur Projekterstellung eingesetzte Hardware anfallen. Durch die ausschließliche Nutzung einer
Entwicklungsumgebung und Werkzeugen aus dem Open-Source-Bereich fallen keine zusätzlichen Kosten wie
Software-Lizenzen an.

\subsection{Entwicklungsprozess}
\label{sec:Entwicklungsprozess}
Der Entwicklungsprozess erfolgt test-driven, was bedeutet dass vor dem Schreiben des eigentichen
Quellcodes einer Funktionalität zunächst der dazugehörige Test entwickelt wird. Diese Vorgehensweise
ermöglicht einen abstrakteren und zielorientierteren Blick auf die eigentlichen geforderten
Funktionen der Applikation, in dem als erstes eine Überlegung über das Ergebnis einer Funktion
und daraufhin die eigentliche Implementierung des jeweiligen Algorithmus erfolgt, welcher zu dem
gewünschten Ergebnis führt.

\subsection{Architekturdesign}
\label{sec:Architekturdesign}
Zur Umsetzung der geforderten Modularität der Applikation bietet sich das sogenannte
\textit{Model-View-Controller-Schema} (MVC) an. Dieses ist ein gängiges Schema beim Aufbau von Webapplikationen und
besteht aus drei Teilen:
\begin{itemize}
	\item das \textit{Model}, welches grob die einzelnen Tabellen einer Datenbank repräsentiert. Es enthält in
	der Regel zusätzliche Logik und Regeln, die zur Verwaltung und Zusammensetzung von Attributen
	eines Datenbankobjektes notwendig sind.
	\item der \textit{View}, welcher eine Ausgabe der gewünschten Daten bereitstellt. Im Kontext einer
	Webapplikation ist dies die eigentlich dargestellte Webseite in HTML/CSS.
	\item der \textit{Controller}, welcher die Schnittstelle zwischen Model und View darstellt.
	Im Controller werden Eingaben verarbeitet und Befehle an das Model weitergegeben, welche
	die angeforderten Informationen an den Controller zurückgibt, der diese wiederum an den
	entsprechenden View verteilt.
\end{itemize}
Diese Aufteilung ermöglicht ein strukturierte und übersichtliche Entwicklung, da Funktionslogik,
Datenbankoperationen und Ausgabelogik klar getrennt und jeweils namentlich zugehörig gekennzeichnet
sind.

TODO: enter figure of MVC schematics

\subsubsection{Wahl der Programmiersprache}
\label{sec:Wahl der Programmiersprache}

\subsubsection{Wahl des Frameworks}
\label{sec:Wahl des Frameworks}


\subsection{Benutzeroberfläche}
\label{sec:Benutzeroberfläche}
\begin{itemize}
	\item Twitter Bootstrap
	\item functional drop-in solution as there's no designer available
\end{itemize}

\subsection{Datenmodell}
\label{sec:Datenmodell}
\begin{itemize}
	\item insert database model chart here?
	\item single table inheritance
	\item tool model for every kind of tool available
\end{itemize}

\subsection{Rechteverteilung}
\label{sec:Rechteverteilung}
\begin{itemize}
	\item general access for regular employee: maintenance of record
	\item admin access for list of all records and API
\end{itemize}

\subsection{Schnittstellen}
\label{sec:Schnittstellen}
\begin{itemize}
	\item API for etsync
\end{itemize}

\subsection{Paketierung}
\label{sec:Paketierung}
\begin{itemize}
	\item RPM
	\item all dependencies included
\end{itemize}
