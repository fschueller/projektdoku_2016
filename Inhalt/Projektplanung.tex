% !TEX root = ../Projektdokumentation.tex
\section{Projektplanung}
\label{sec:Projektplanung}


\subsection{Projektphasen}
\label{sec:Projektphasen}
Für das Projekt wurde folgende Zeitplanung aufgestellt:

\tabelle{Zeitplanung}{tab:Zeitplanung}{ZeitplanungKomplett}

\subsection{Ressourcenplanung}
\label{sec:Ressourcenplanung}
Die Planung der benötigten Ressourcen untergliedert sich in Personal- und Kostenplanung.
\subsubsection{Personalplanung}
\label{sec:Personalplanung}
Außer dem Auszubildenden werden keine weiteren Mitarbeiter für das Projekt eingesetzt. Dieser wird
dafür von seiner regulären Mitarbeit im Team freigestellt.
\subsubsection{Kostenplanung}
\label{sec:Kostenplanung}
Die Personalkosten belaufen sich auf die Ausbildungsvergütung für den Zeitraum des Projekts.

Als operative Kosten fallen lediglich die Stromkosten für die zur Projekterstellung eingesetzte
Hardware an. Es wird keine zusätzliche Hardware zur Entwicklung benötigt.
Durch die ausschließliche Nutzung einer Entwicklungsumgebung und Werkzeugen aus dem
Open-Source-Bereich entstehen keine zusätzlichen Kosten wie Software-Lizenzen.

\subsection{Entwicklungsprozess}
\label{sec:Entwicklungsprozess}
Als Entwicklungsmethode wird \textit{Test Driven Development} (\acs{TDD}) eingesetzt. Diese Methode
sieht vor, dass vor dem Schreiben des eigentichen Quellcodes einer Funktionalität zunächst der
dazugehörige Test entwickelt wird. Dadurch wird ein einen abstrakterer und
zielorientierterer Blick auf die eigentlichen geforderten Funktionen der Applikation ermöglicht, in
dem als erstes eine Überlegung über das Ergebnis einer Funktion und daraufhin die eigentliche Implementierung
des jeweiligen Algorithmus erfolgt, welcher zu dem gewünschten Ergebnis führt.

\subsection{Architekturdesign}
\label{sec:Architekturdesign}
Zur Umsetzung der geforderten Modularität der Applikation wurde das sogenannte
\textit{Model-View-Controller-Schema} (\acs{MVC}) verwendet. Dieses ist ein gängiges Schema beim Aufbau von
Webapplikationen und besteht aus drei Teilen:
\begin{itemize}
	\item das \textit{Model}, welches grob die einzelnen Tabellen einer Datenbank repräsentiert. Es enthält in
	der Regel zusätzliche Logik und Regeln, die zur Verwaltung und Zusammensetzung von Attributen
	eines Datenbankobjektes notwendig sind.
	\item der \textit{View}, welcher eine Ausgabe der gewünschten Daten bereitstellt. Im Kontext einer
	Webapplikation ist dies die eigentlich dargestellte Webseite in \textit{HTML/CSS}.
	\item der \textit{Controller}, welcher die Schnittstelle zwischen Model und View darstellt.
	Im Controller werden Eingaben verarbeitet und Befehle an das Model weitergegeben, welche
	die angeforderten Informationen an den Controller zurückgibt, der diese wiederum an den
	entsprechenden View verteilt.
\end{itemize}
Diese Aufteilung ermöglicht ein strukturierte und übersichtliche Entwicklung, da Funktionslogik,
Datenbankoperationen und Ausgabelogik klar getrennt und jeweils namentlich zugehörig gekennzeichnet
sind.

\abb{MVC-Schema}{fig:MVC-Schema}{MVC-Model.pdf}{0.5}

\subsection{Wahl der Programmiersprache}
\label{sec:Wahl der Programmiersprache}
Die objektorientierte Programmiersprache \textit{Ruby} legt Wert auf die Balance von \textit{funktionaler} und
\textit{imperativer Programmierung}. So verzichtet Ruby beispielsweise auf Klammern zur Kennzeichnung
von Codeblöcken und behandelt jeden Bestandteil des Codes, wie Variablen, als ein \textit{Objekt},
welches eigene Methoden besitzt. Das macht die Verarbeitung und Manipulation von Informationen sehr
intuitiv und leicht nachvollziehbar. Zusätzlich werden Ruby und sein Web-Framework \textit{Ruby on Rails}
quelloffen verwaltet, entwickelt und sind gut dokumentiert. Unter Berücksichtigung dieser Aspekte
wurde Ruby und \acs{RoR} als Programmiersprache für das Projekt gewählt.

Neben den existierenden Programmbibliotheken existieren zusätzlich von der Community bereitgestellte Module,
sogenannte \textit{Gems}. Die Einbindung dieser Gems in ein bestehendes Projekt vermeidet
Duplikation und erspart viel Entwicklungszeit.

\subsection{Benutzeroberfläche}
\label{sec:Benutzeroberfläche}
Da nicht auf Mitarbeiter aus dem Team der Webdesigner zurückgegriffen und möglichst zeiteffizient
eine für Nutzer verständliche Oberfläche realisiert werden musste, fiel die Entscheidung auf
\textit{Twitter Bootstrap}.
Das CSS-Framework Twitter Bootstrap wird häufig zur Gestaltung von Webseiten und
-applikationen verwendet, da es ein über eine umfangreiche Bibliothek an Gestaltungsvorlagen
verfügt und sehr leicht in ein Projekt zu integrieren ist. Auch unerfahrenen Anwendern gelingt damit
mühelos ein ansprechende und responsive Nutzeroberfläche.

\subsection{Rechteverteilung}
\label{sec:Rechteverteilung}
Die Applikation soll von zwei verschiedenen Arten von Usern verwendet werden: dem regulären
Mitarbeiter, der einzig Zugriff auf seine verwalteten Informationen zu externen Services hat, und
einem Administrator, der neben der Bearbeitung seines eigenen Datensatzes auch jene aller in der
Datenbank vorhandenen Mitarbeiter einsehen, abfragen und löschen kann.

Diese Rechteverteilung kann über ein \textit{Flag} gelöst werden, was die einfachste Methode ist.
Sie ist allerdings aus sicherheitsrelevanter Sicht sehr anfällig und leicht zu manipulieren,
weswegen entschieden wurde, die Rechteverteilung auf Datenbankebene zu realisieren.

\subsection{Datenmodell}
\label{sec:Datenmodell}
Das Projekt verfügt über zwei Informationsträger, die in der Datenbank abgebildet werden
sollen: den \textit{User}-Objekten und den jeweils zugehörigen \textit{Tool}-Objekten.

\abb{Entity-Relationship-Modell}{fig:ER-Modell}{ER-Diagramm.pdf}{0.4}

Da für die Applikation zusätzliche Spezialisierungen, wie eine Administrator-Rolle und verschiedene
externe Services, notwendig sind, wurden diese Beziehungen mit Hilfe der sogenannten
\textit{Single Table Inheritance} (\acs{STI}) realisiert.

\abb{Beispielhafte Darstellung der STI}{fig:STI-Beispiel}{STI-Figure.pdf}{0.5}

Diese spezielle Vererbungsstrategie lässt sich mit Hilfe der in Rails eingebauten
\textit{Active Record}-Engine durch das einfache Hinzufügen eines Typ-Attributs an das Model
umsetzen.

\subsection{Schnittstellen}
\label{sec:Schnittstellen}
Die Hauptaufgabe der Applikation ist die Bereitstellung der gesammelten Informationen zu den
Mitarbeitern und deren eingetragenen Informationen zu ihren genutzten Services. Dazu soll eine
eigene Schnittstelle innerhalb der Applikation entwickelt werden, welche die Informationen im
\textit{JSON}-Format ausgibt. Diese Schnittstelle kann nur authentifiziert angesprochen werden,
\dahe der User muss eine Administratorrolle besitzen, um Zugriff zu erhalten.

\abb{Visualisierung der internen Schnittstelle}{fig:etsync-api}{etsync-api.pdf}{0.7}

\subsection{Paketierung}
\label{sec:Paketierung}
Gemäß der Richtline des Teams, in dem die Applikation entwickelt wurde, wird jene als \textit{\acs{RPM}}
paketiert. Dies ermöglicht eine einfache und sofort funktionsfähige Installation, da alle
notwendigen Programmabhängigkeiten im Paket definiert und während des Installationsprozesses
abgerufen und mitinstalliert werden. Dadurch ist beispielsweise auch bei Bedarf eine problemlose
Migration auf einen anderen Server möglich.
