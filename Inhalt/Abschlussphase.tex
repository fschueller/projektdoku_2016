% !TEX root = ../Projektdokumentation.tex
\section{Projektabschluss}
\label{sec:Projektabschluss}

\subsection{Vergleich der Zeitplanung}
\label{sec:Vergleich der Zeitplanung}

Beinahe alle Projektphasen konnten im geplanten Zeitraum ausgeführt werden. Lediglich bei
der Anbindung externer Schnittstellen wurde mehr Zeit benötigt, da die Testentwicklung der
\acs{LDAP}-Verbindung aufwändiger war als zuvor angenommen. Auch die Implementierung eines
effizienten Algorithmus für die \Code{LDAPSearch}-Klasse erwies sich als zeitintensiver als geplant.
Die komplette Paketierung samt aller Abhängigkeiten nahm auch marginal mehr Zeit in Anspruch, da
nicht alle Gems in der gewünschten Version paketiert vorlagen.

Diese Differenz konnte allerdings durch die stark verkürzte Testphase ausgeglichen werden, welche im
Abschnitt Testphase (\Vgl \ref{sec:Testphase}) näher ausgeführt ist.

\tabelle{Soll-/Ist-Vergleich}{tab:Soll-/Ist-Vergleich}{Zeitnachher}
\pagebreak

\subsection{Soll-/Ist-Vergleich}
\label{sec:SollIstVergleich}
Zum Abschluss des Projektes erfolgt eine Kontrolle, dass alle im Soll-Konzept aufgeführten
Anforderungen erfüllt werden:
\begin{itemize}
  \item Zusammenfassend kann angeführt werden, dass eine Webapplikation im Stil eines Self-Service
  geschaffen wurde. Diese wurde mit Ruby sowie Ruby on Rails in einem modularen Aufbau erstellt.
  \item Als Datenquelle für die Mitarbeiterinformationen wird das interne eDirectory verwendet.
  \item Über die realisierten Benutzerrollen ist die Verwaltung der eigenen Informationen für jeden Benutzer
  möglich. Jedoch ist nur Administratoren zusätzlich Einsicht in die Datenbank sowie das Entfernen von
  Benutzern aus dem Tool gestattet.
  \item Die API ist nur als authentifizierter Administrator benutzbar.
\end{itemize}

\subsection{Projektabnahme}
\label{sec:Projektabnahme}
Die Projektabnahme erfolgt mit einer kurzen Präsentation und Demonstration der Applikation für den
Projektbetreuer sowie die in Frage kommenden Administratoren der Organisationen.

Bereits während dieser Phase war eine deutliche Verbesserung gegenüber dem in der Ist-Analyse
aufgezeigten Prozesses zu vermerken.

\subsection{Firmenweite Einführung}
\label{Firmenweite Einführung}
Die firmenweiter Einführung des Tools erfolgte in mehreren Schritten:

\begin{enumerate}
  \item Der operative Betrieb der Applikation sowie die Freigabe und Bekanntgabe wird durch SUSE-IT
  sichergestellt.
  \item Zu einem festgelegten Termin ist eine E-Mail mit einer Anleitung für die Benutzer versendet worden.
  \item Nach der Einführungsphase ist das Tool in die Firmenrichtlinie \anf{Nutzung interner Entwicklungstools}
  aufgenommen worden.
\end{enumerate}
\pagebreak
