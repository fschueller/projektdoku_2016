% !TEX root = ../Projektdokumentation.tex
\section{Projektabschluss}
\label{sec:Projektabschluss}

\subsection{Vergleich der Zeitplanung}
\label{sec:Vergleich der Zeitplanung}

Beinahe alle Projektphasen konnten im geplanten Zeitraum ausgeführt werden. Lediglich bei
der Anbindung externer Schnittstellen wurde mehr Zeit benötigt, da die Testentwicklung der
\acs{LDAP}-Verbindung aufwändiger war als zuvor angenommen. Auch die Implementierung eines
effizienten Algorithmus für die \Code{LDAPSearch}-Klasse erwies sich als zeitintensiver als geplant.
Die komplette Paketierung samt aller Abhängigkeiten nahm auch marginal mehr Zeit in Anspruch, da
nicht alle Gems in der gewünschten Version paketiert vorlagen.

Diese Differenz konnte allerdings durch die stark verkürzte Testphase ausgeglichen werden.

\tabelle{Soll-/Ist-Vergleich}{tab:Soll-/Ist-Vergleich}{Zeitnachher}
\pagebreak

\subsection{Soll-/Ist-Vergleich}
\label{sec:SollIstVergleich}

\subsection{Projektabnahme}
\label{sec:Projektabnahme}
Die Projektabnahme erfolgte mit einer kurzen Präsentation und Demonstration der Applikation für den
Projektbetreuer sowie die in Frage kommenden Administratoren der Organisationen.

\subsection{Firmenweite Einführung}
\label{Firmenweite Einführung}
Nach Abnahme des Projekts wurde zu einem festgelegten Stichtag eine Ankündigungsmail auf den
internen Mailinglisten verschickt, welche kurz den Nutzen der Applikation erklärt und alle
Mitarbeiter zur Eintragung ihrer Daten aufruft. Desweiteren wurde das Anlegen des Datensatzes für
jeden neuen Mitarbeiter verpflichtend als Richtlinie eingeführt.
