% !TEX root = ../Projektdokumentation.tex
\section{Implementierungsphase}
\label{sec:Implementierungsphase}

\subsection{Einleitung}
\label{sec:Einleitung}
Nach der Planungsphase wurde im direkten Anschluss die Implementierungsphase mit folgenden Schritten
eingeleitet:
\begin{enumerate}
	\item Implementierung des Backends
	\item Implementierung der Benutzeroberfläche
	\item Anbindung externer Schnittstellen
	\item Implementierung der internen Schnittstelle
	\item Durchführung einer Testphase
	\item Paketierung der Applikation
	\item Deployment auf Produktivmaschine
\end{enumerate}

\subsection{Implementierung des Backends}
\label{sec:Implementierung des Backends}
Die Entwicklung der Applikation begann mit der initialen Erstellung eines \acs{RoR}-Projektes.
Nach Installation der zugehörigen Entwicklungsumgebung geschieht dies einfach mit Ausführung des
Befehls \Code{rails new} in dem gewünschten Verzeichnis. Mit Hilfe dieses Befehls wird eine
standardisierte Projektmappe erzeugt.

Hier sei anzumerken dass RoR viele eingebaute Befehle mitbringt, welche die Ausführung von
Entwicklungsschritten, wie \bspw das Anlegen von bestimmten Klassen, automatisiert. Ein manuelles
Anlegen von relevanten Dateien im korrekten Verzeichnis entfällt dadurch und der Entwicklungsprozess
kann massiv beschleunigt werden.

Das \textit{Gemfile} bildet die zentrale Auflistung der von der Applikation genutzten Gems. Da die
Entwicklung der Applikation im Sinne des \acs{TDD} (\Vgl \ref{sec:Entwicklungsprozess}) erfolgt,
wurde hier zunächst die Testsuite \textit{rspec} hinzugefügt und die ersten Tests für die Models
(\Vgl \ref{sec:Architekturdesign}) geschrieben.

\subsubsection{Erstellen der Models}
\label{sec:Erstellen der Models}
Erst nach Schreiben der Tests erfolgte die Erstellung der eigentlichen Modeldateien. samt jeweils
zugehörigen Attributen. Dies geschieht mit Hilfe sogenannter \textit{Migrationen},
Datenbankoperationen geschrieben in Ruby. Zum Hinzufügen der so definierten Tabellen und Attribute
zur Datenbank wird der Befehl \Code{rake db:migrate} ausgeführt, welcher die Ruby-Syntax in
\textit{\acs{SQL}} übersetzt.

Um die Integrität der Datenbanktabellen zu erhalten und \ggfs Falscheingaben des Anwenders abzufangen,
werden in den Models Validierungen implementiert. Diese werden bei jeder Datenbankoperation
ausgeführt und verhindern bei Eingabe fehlerhafter Daten ein Abspeichern in der Datenbank.

TODO: insert code example of user model validation/tool model validation

Im Anschluss dessen wurde eine weitere \textit{Migration} ausgeführt, welche ein Typ-Attribut zu
beiden Models hinzufügt. Dadurch wurde die in der Projektplanung niedergelegte Vererbungsstrategie
der \acs{STI} realisiert (\Vgl \ref{sec:Datenmodell}).

\subsubsection{Erstellen der Controller}
\label{sec:Erstellen der Controller}
Wie auch bei den Models erfolgt vor Erstellung der Controller und während der Implementierung
das Schreiben von Tests, in denen die grundlegend erwarteten Verhaltensweisen der Controller
getestet werden.

Den Einstiegspunkt der Applikation bildet der \textit{Application Controller}. In diesem wurden
verschiedene Strategien implementiert, welche den Betrieb der gesamten Applikation betreffen,
\bspw das initiale Aufrufen der Nutzerauthentifizierung. Alle anderen Controller erben von diesem
Application Controller und verfügen damit bei Aufruf über alle dort niedergelegten Methoden.

Per Konvention wird für jedes Model mindestens ein Controller erstellt, jedoch benötigt nicht jeder
Controller ein Model. In jedem Controller befinden sich sogenannte \textit{Actions}. Ruft der Nutzer
\bspw die Webseite zum Hinzufügen eines neuen Alias für einen externen Service auf, wird seine
Anfrage an die \Code{new}-Action des jeweiligen Controllers weitergeleitet
(\Vgl \ref{sec:Architekturdesign}).

Zunächst wird der \textit{ExternaltoolsController} erstellt. Alle relevanten Operationen
wie Anzeige, Hinzufügen und Löschen von Daten des Tool-Models werden von diesem Controller
verarbeitet. Dazu fragt der Controller Informationen zu dem eingeloggten Nutzer aus der User-Tabelle
ab, welches über das Objekt \Code{current user} realisiert wird. Dieses Objekt stammt aus dem
eingesetzten Authentifizierungs-Gem \textit{Devise} (\Vgl \ref{sec:Anbindung externer Schnittstellen}).

TODO: code example of externaltools controller

Anschließend werden in einem Unterverzeichnis die Controller für den Administratorbereich der
Applikation angelegt. Um diese vor einem unberechtigten Aufruf zu schützen, wird ein
\textit{AdminController} erstellt, dessen einzige Methode prüft, ob ein Administrator eingeloggt ist.
Ist dies nicht der Fall, erfolgt eine Weiterleitung zur Index-Seite.

Da nur die Administratoren die Möglichkeit besitzen sollen, Nutzer aus der Datenbank zu entfernen,
befindet sich der \textit{UsersController} in diesem Unterverzeichnis. Als Kernstück dieses
Controllers wird die \Code{list user}-Action implementiert, welche die gesammelten Informationen
zu allen Mitarbeitern aus der Datenbank abfragt und die API bereitstellt.

TODO: code example of users controller

In ähnlicher Weise wird der \textit{UnderlingsController} implementiert, welcher aber nur
Informationen zu Teammitgliedern abfragt.

\subsection{Implementierung der Benutzeroberfläche}
\label{sec:Implementierung der Benutzeroberfläche}
Nach der Fertigstellung des grundlegenden Backends erfolgt die Programmierung der Benutzeroberfläche.
Auch für diese werden Tests geschrieben, sogenannte \textit{Feature Tests}.

TODO: code example of feature test

Wie in der Planung (\Vgl \ref{sec:Benutzeroberfläche}) niedergelegt, wird für die Implementierung
das Framework Twitter Bootstrap verwendet. Die Einbindung in das Projekt erfolgt als \textit{Gem}.

Die gewünschte Darstellung als \textit{Dashboard} mit Paneel-Elementen lässt sich zeitsparend mit
den in Bootstrap vorhandenen Vorlagen erstellen. Die Farbgebung der Elemente wird gemäß der
Firmenrichtlinien gestaltet.

TODO: Screenshot basic externaltools view

Als spezielle Strategie zur Verbesserung der Performanz der Applikation werden sogenannte
\textit{Partials} eingesetzt. Bei dieser Vorgehensweise werden bei Webseiten bei Bedarf nur die
sich verändernden Elemente neu geladen, statt die komplette Seite neu abzurufen.
Dies verringert die benötigte Antwortzeit des Webservers maßgeblich.

TODO: Screenshot externaltools new action view

Für Zugang zu den Administratoransichten wird die reguläre Mitarbeiteransicht um weitere Tab-Elemente
erweitert. Diese sind nur bei Login als Administrator sichtbar.

TODO: Screenshot admin view

\subsection{Anbindung externer Schnittstellen}
\label{sec:Anbindung externer Schnittstellen}
Wie bei jedem Entwicklungsschritt beginnt die Anbindung von externen Schnittstellen mit dem Schreiben
von Tests.

Als Authentifizierungsmethode wird eine Authentifizierung über \textit{\acs{LDAP}} gewählt. Darüber
kann das firmeneigene eDirectory zum Abgleich genutzt und gleichzeitig Informationen über
den Mitarbeiter abgerufen werden.

Das Gem \textit{Devise} bietet eine umfassende Sammlung an Konfigurationsmöglichkeiten und wird
daher als Lösung eingesetzt. Beim erstmaligen Einloggen eines Mitarbeiters werden gleichzeitig
dessen Name, Mitarbeiterkürzel und Informationen zu seinem Standort sowie der internen
e-Mail-Adresse in der Datenbank abgespeichert. Informationen zu seinem Vorgesetzten werden einzeln
dynamisch bei Bedarf abgefragt und angezeigt, da sich dieses Attribut öfter ändern kann.

Ebenso soll der Mitarbeiterstatus möglichst aktuell als Information verfügbar sein. Da diese
Information allein in der Administratoransicht \bzw bei Ansprechen der Schnittstelle abgefragt wird,
würde eine gesammelte Abfrage des Status für mehrere hundert Mitarbeiter gleichzeitig den Webserver
überlasten. Aus diesem Grund wird ein Hintergrundprozess implementiert. RoR bietet dafür den
sogenannten \textit{Active Job}-Mechanismus an. Dort wird ein Prozess als Klasse implementiert,
welcher einmal pro Woche für alle eingetragenen Mitarbeiter einzeln den Mitarbeiterstatus anfragt
und als \textit{boolean flag} in der Datenbank aktualisiert.

TODO: code example employee status job

Als Kernstück wird die Klasse \Code{LDAPSearch} implementiert, welche für alle zusätzlichen
LDAP-Abfragen außerhalb des User-Models verwendet wird.

TODO: code example ldap search.rb

Gewünscht war ebenfalls die Möglichkeit, dass Mitarbeiter \ggfs in Eigeninitiative eine
Beitrittsanfrage an die Administratoren der Organisationen stellen können. Zu diesem Zweck wird der
Status der Organisationszugehörigkeit ebenfalls durch einen ActiveJob bei der API von Github
angefragt, welcher beim Hinzufügen eines Github-Nutzernamens ausgelöst wird.
Bei nicht vorhandener Zugehörigkeit wird ein Link bereitgestellt, welcher eine fertige e-Mail-Vorlage
an die Administratoren generiert.

Aus Zeitmangel konnte eine ähnliche Strategie für den externen Service Trello während des
Projektverlaufs nicht mehr implementiert werden. Da dies aber nicht als kritisch bewertet wurde,
behinderte dies nicht den Abschluss des Projektes.

\subsection{Interne Schnittstelle}
\label{sec:Interne Schnittstelle}
Die Progammierung der Schnittstelle begann zunächst mit der Analyse des Kommandozeilenwerkzeugs
\textit{etsync}, für welches die Schnittstelle entwickelt werden soll. Anhand des Codes in etsync
wird die Ausgabe der Schnittstelle modelliert. Realisiert wird die Ausgabe in \textit{\acs{JSON}},
was ein gebräuchliches Datenformat zum Austausch von Informationen zwischen Anwendungen ist.

Um die gewünschten Informationen korrekt und einfach zu modellieren, wird das Gem \textit{rabl}
eingesetzt. Mit diesem Gem kann die JSON-Notation einfach als Active Record-Anfragen in Ruby
geschrieben werden und vermeidet so Fehler bei der sonst manuellen Erstellung einer JSON-Vorlage,
welche viele Klammern und Satzzeichen verwendet. Auch können spätere Anpassungen so leichter
durchgeführt werden und eine genaue Kenntnis der JSON-Syntax ist überflüssig.

Die zugehörige Controller-Action befindet sich im Administratorverzeichnis, weshalb sie nur von
Administratoren authentifiziert angesprochen werden kann (\Vgl \ref{sec:Erstellen der Controller}).

\subsection{Testphase}
\label{sec:Testphase}
Nach Implementierung des Projektes konnte eine finale Testphase zeitlich gering gehalten werden,
da die Applikation während der Entwicklung bereits durch das Prinzip des \acs{TDD} laufend
getestet und Fehler behoben wurden. Es wurden neben einem abschließenden kompletten Durchlauf der
Testsammlung \textit{Usability-Tests} der Nutzeroberfläche mit Probanden und realen Mitarbeiterdaten
durchgeführt.

\subsection{Paketierung der Applikation}
\label{sec:Paketierung der Applikation}
Wie bereits in \ref{sec:Paketierung} dargelegt erfolgt nach Entwicklung und finaler Testphase der
Applikation die Paketierung als RPM. Die Kontrollstruktur eines RPM-Pakets ist das sogenannte
\textit{Specfile}. In diesem werden alle relevanten Informationen zu dem Softwarepaket,
wie \bspw Versionsnummer oder Abhängigkeiten auf andere Pakete, niedergelegt.

Um das Paket immer auf dem neuesten Stand zu halten, wird über ein Skript bei einer
Codeaktualisierung des Projekts auf der firmeneigenen Versionsverwaltungsplattform automatisch die
Ausführung der Testsammlung ausgelöst. Sind diese erfolgreich mit den neuesten Änderungen ausgeführt
worden, wird die Paketierung automatisiert angestoßen. Dies verhindert zusätzlich, dass fehlerhafte
oder ungetestete Funktionalität in das endgültige Paket gelangt.

\subsection{Deployment}
\label{sec:Deployment}
Die Bereitstellung der Applikation geschieht in Kollaboration mit dem Team, in welchem das
Projekt durchgeführt wurde. Auf einer dedizierten Produktivmaschine, die bereits als Webserver für
andere Applikationen fungiert, wird eine weitere Konfiguration angelegt und das Software-Paket der
Applikation installiert. Nach Neustart der zugehörigen Services und Prozesse ist die
Webapplikation intern über den Browser zu erreichen.
