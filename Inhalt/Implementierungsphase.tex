% !TEX root = ../Projektdokumentation.tex
\lstset{
  backgroundcolor=\color{white}
}

\section{Implementierungsphase}
\label{sec:Implementierungsphase}

\subsection{Einleitung}
\label{sec:Einleitung}
Nach der Planungsphase wurde im direkten Anschluss die Implementierungsphase eingeleitet. Diese
umfasst die folgenden Schritte, welche in diesem Kapitel einzeln näher erläutert werden:
\begin{enumerate}
	\item Implementierung des Backends
	\item Implementierung der Benutzeroberfläche
	\item Anbindung externer Schnittstellen
	\item Implementierung der internen Schnittstelle
	\item Durchführung einer Testphase
	\item Paketierung der Applikation
	\item Deployment auf Produktivmaschine
\end{enumerate}

\subsection{Implementierung des Backends}
\label{sec:Implementierung des Backends}
Als erster Schritt erfolgt die initiale Erstellung eines \acs{RoR}-Projektes.
Nach Installation der zugehörigen Entwicklungsumgebung geschieht dies einfach mit Ausführung des
Befehls \Code{rails new} im Arbeitsverzeichnis. Mit Hilfe dieses Befehls wird eine
standardisierte Projektmappe erzeugt.

RoR bringt viele eingebaute Kommandozeilenbefehle mit, welche die Ausführung von
Entwicklungsschritten automatisiert. Beispielsweise kann das Anlegen von relevanten Dateien im
korrekten Verzeichnis direkt über einen solchen Befehl erledigt werden. Dadurch beschleunigt sich der Entwicklungsprozess massiv.

Das \textit{Gemfile} bildet die zentrale Auflistung der von der Applikation genutzten Gems. Da die
Entwicklung der Applikation im Sinne des \acs{TDD} (\Vgl \ref{sec:Entwicklungsprozess}) erfolgt,
wurde hier zunächst die Testsuite \textit{rspec} hinzugefügt und die ersten Tests für die Models
(\Vgl \ref{sec:Architekturdesign}) geschrieben.

\subsubsection{Erstellen der Models}
\label{sec:Erstellen der Models}
Erst nach Schreiben der Tests erfolgte die Erstellung der eigentlichen Modeldateien samt jeweils
zugehörigen Attributen. Dies geschieht mit Hilfe sogenannter \textit{Migrationen},
Datenbankoperationen geschrieben in Ruby. Zum Hinzufügen der so definierten Tabellen und Attribute
zur Datenbank wird der Befehl \Code{rake db:migrate} ausgeführt, welcher die Ruby-Syntax in
\textit{\acs{SQL}} übersetzt.

Um die Integrität der Datenbanktabellen zu erhalten und \ggfs Falscheingaben des Anwenders abzufangen,
werden in den Models Validierungen implementiert. Diese werden bei jeder Datenbankoperation
ausgeführt und verhindern bei Eingabe fehlerhafter Daten ein Abspeichern in der Datenbank.

\snip{externaltool.rb}{1}{6}

Im Anschluss dessen wurde eine weitere Migration ausgeführt, welche ein Typ-Attribut zu
beiden Models hinzufügt. Dadurch wurde die in der Projektplanung niedergelegte Vererbungsstrategie
der \acs{STI} realisiert (\Vgl \ref{sec:Datenmodell}).

\subsubsection{Erstellen der Controller}
\label{sec:Erstellen der Controller}
Per Konvention wird für jedes Model mindestens ein Controller erstellt, allerdings benötigt nicht jeder
Controller ein Model. Jedoch befinden sich in jedem Controller eine oder mehr Methoden, sogenannte
Actions. Ruft ein Nutzer beispielsweise die Webseite zum Hinzufügen eines neuen Alias für einen
externen Service auf, wird seine Anfrage an die \Code{new}-Action des jeweiligen Controllers weitergeleitet
(\Vgl \ref{sec:Architekturdesign}).

Den Einstiegspunkt der Applikation bildet der \anf{Application Controller}. In diesem wurden
verschiedene Strategien implementiert, welche den Betrieb der gesamten Applikation betreffen,
\bspw das initiale Aufrufen der Nutzerauthentifizierung. Alle anderen Controller erben von diesem
Application Controller und verfügen damit bei Aufruf über alle dort niedergelegten Methoden.

Zunächst wird der \anf{Externaltools Controller} erstellt. Alle relevanten Operationen
wie Anzeige, Hinzufügen und Löschen von Daten des Tool-Models werden von diesem Controller
verarbeitet. Dazu fragt der Controller Informationen zu dem eingeloggten Nutzer aus der User-Tabelle
ab. Dies wird über das Objekt \Code{current user} realisiert. Das Objekt stammt aus dem
eingesetzten Authentifizierungs-Gem \anf{Devise} (\Vgl \ref{sec:Anbindung externer Schnittstellen}).

\snip{externaltools.controller.rb}{18}{33}

Zur Abgrenzung des Administratorbereichs der Applikation wird ein Administratorverzeichnis angelegt.
Um die Controller der dort zu implementierenden Funktionen vor einem unberechtigten Zugriff zu schützen, wird ein
\anf{Admin Controller} erstellt. Dessen einzige Methode prüft, ob ein Administrator eingeloggt ist.
Ist dies nicht der Fall, erfolgt eine Weiterleitung zur Index-Seite.

Da nur die Administratoren die Möglichkeit besitzen sollen, Nutzer aus der Datenbank zu entfernen,
befindet sich der \anf{Users Controller} in diesem Unterverzeichnis. Als Kernstück dieses
Controllers wird die \Code{list user}-Action implementiert, welche die gesammelten Informationen
zu allen Mitarbeitern aus der Datenbank abfragt und die API bereitstellt.

\snip{users.controller.rb}{26}{29}

In ähnlicher Weise wird der \anf{Underlings Controller} implementiert, welcher aber nur
Informationen zu Teammitgliedern abfragt.

\subsection{Implementierung der Benutzeroberfläche}
\label{sec:Implementierung der Benutzeroberfläche}
Nach der Fertigstellung des grundlegenden Backends erfolgt die Programmierung der Benutzeroberfläche.
Auch für diese werden Tests geschrieben, die sogenannten \textit{Feature Tests}.

\snip{employee.signin.spec.rb}{11}{15}

Wie in der Projektplanung (\Vgl \ref{sec:Benutzeroberfläche}) niedergelegt, wird für die Implementierung
das Framework Twitter Bootstrap verwendet.

Die gewünschte Darstellung als \textit{Dashboard} (\Vgl \ref{sec:Soll-Konzept}) mit Panel-Elementen
lässt sich zeitsparend mit den in Bootstrap vorhandenen Vorlagen erstellen. Die Farbgebung der Elemente wird gemäß der
Firmenrichtlinien gestaltet.

\abb{Nutzeroberfläche}{fig:Nutzeroberfläche}{externaltoolsscreen.pdf}{0.7}

Als Strategie zur Verbesserung der Performance der Applikation werden sogenannte
\textit{Partials} eingesetzt. Bei dieser Herangehensweise werden bei Bedarf nur die sich
verändernden Elemente auf einer Webseite neu geladen, statt die komplette Seite neu abzurufen.
Dies verringert die benötigte Antwortzeit des Webservers maßgeblich.

Für Zugang zu den Administratoransichten wird die reguläre Mitarbeiteransicht um weitere Tab-Elemente
erweitert. Diese sind nur bei Login als Administrator sichtbar.

\subsection{Anbindung externer Schnittstellen}
\label{sec:Anbindung externer Schnittstellen}
Als Authentifizierungsmethode wird eine Authentifizierung über \textit{\acs{LDAP}} gewählt. Damit
kann das firmeneigene eDirectory zum Abgleich genutzt und gleichzeitig Informationen über
den Mitarbeiter abgerufen werden.

Das Gem \textit{Devise} bietet eine umfassende Sammlung an Konfigurationsmöglichkeiten und wird
daher als Lösung eingesetzt. Beim erstmaligen Einloggen eines Mitarbeiters werden gleichzeitig
dessen Name, Mitarbeiterkürzel und Informationen zu seinem Standort sowie die interne
e-Mail-Adresse in der Datenbank abgespeichert. Informationen zum jeweiligen Vorgesetzten des
Nutzers werden einzeln bei Bedarf abgefragt und angezeigt.

Der Beschäftigungsstatus eines Mitarbeiters soll möglichst aktuell als Information vorliegen. Da diese
Information allein in der Administratoransicht \bzw über die Schnittstelle abgefragt wird,
würde eine gesammelte Abfrage des Status für mehrere hundert Mitarbeiter gleichzeitig den LDAP-Server
überlasten. Aus diesem Grund wird ein Hintergrundprozess implementiert. RoR bietet dafür den
sogenannten \textit{Active Job}-Mechanismus an. Dort wird ein Prozess als Klasse implementiert,
welcher einmal pro Woche für alle eingetragenen Mitarbeiter einzeln den Mitarbeiterstatus anfragt
und als \textit{boolean flag} in der Datenbank aktualisiert.

\snip{employee.status.job.rb}{1}{13}

Als Kernstück wird die Klasse \Code{LDAPSearch} implementiert, welche für alle zusätzlichen
LDAP-Abfragen außerhalb des User-Models verwendet wird.

\snip{ldap.search.rb}{4}{14}

Gewünscht war ebenfalls die Möglichkeit, dass Mitarbeiter \ggfs in Eigeninitiative eine
Beitrittsanfrage an die Administratoren der Organisationen stellen können. Zu diesem Zweck wird der
Status der Organisationszugehörigkeit ebenfalls durch einen ActiveJob bei der API von Github
angefragt, welcher beim Hinzufügen eines Github-Nutzernamens ausgelöst wird.
Bei nicht vorhandener Zugehörigkeit wird ein Link bereitgestellt, welcher eine fertige e-Mail-Vorlage
an die Administratoren generiert.

Aus Zeitmangel konnte eine ähnliche Strategie für den externen Service Trello während des
Projektverlaufs nicht mehr implementiert werden. Da dies aber nicht als kritisch bewertet wurde,
behinderte dies nicht den Abschluss des Projektes.

\subsection{Interne Schnittstelle}
\label{sec:Interne Schnittstelle}
Die Schnittstelle wird für die Verarbeitung durch das Kommandozeilenwerkzeug \anf{etsync} entwickelt.
Anhand des bestehenden Codes in \anf{etsync} wird die Ausgabe der Schnittstelle modelliert.
Realisiert wird die Ausgabe in \textit{\acs{JSON}},
welches ein gebräuchliches Datenformat zum Austausch von Informationen zwischen Anwendungen ist.

Um die gewünschten Informationen korrekt und einfach zu modellieren, wird das Gem \textit{rabl}
eingesetzt. Mit diesem Gem kann die JSON-Notation einfach als Datenbankabfragen in Ruby
geschrieben werden und vermeidet so Fehler bei der sonst manuellen Erstellung einer JSON-Vorlage,
welche viele Klammern und Satzzeichen verwendet. Auch können spätere Anpassungen so leichter
durchgeführt werden da eine genaue Kenntnis der JSON-Syntax überflüssig ist.

Die zugehörige Controller-Action befindet sich im Administratorverzeichnis, weshalb auf die Schnittstelle
nur von Administratoren authentifiziert zugegriffen werden kann (\Vgl \ref{sec:Erstellen der Controller}).

\subsection{Testphase}
\label{sec:Testphase}
Nach Implementierung des Projektes konnte eine finale Testphase zeitlich gering gehalten werden.
Durch den Einsatz von \acs{TDD} wurde die Applikation während der Entwicklung bereits laufend
getestet und eventuelle Fehler wurden behoben. Neben der Durchführung von \textit{Usability-Tests}
der Nutzeroberfläche mit Probanden und realen Mitarbeiterdaten fand abschließend ein kompletter
Durchlauf der Testsammlung statt.

\subsection{Paketierung der Applikation}
\label{sec:Paketierung der Applikation}
Wie bereits in der Projektplanung dargelegt erfolgt nach Entwicklung und finaler Testphase der
Applikation die Paketierung als RPM. Die Kontrollstruktur eines RPM-Pakets ist das sogenannte
\textit{Specfile}. In diesem werden alle relevanten Informationen zu dem Softwarepaket,
wie \bspw Versionsnummer oder Abhängigkeiten auf andere Pakete, niedergelegt.

Um das Paket immer auf dem neuesten Stand zu halten, wird über ein Skript bei einer
Codeaktualisierung des Projekts automatisch die Ausführung der Testsammlung ausgelöst.
Sind diese erfolgreich mit den neuesten Änderungen ausgeführt worden, wird die Paketierung
automatisiert angestoßen. Dies verhindert zusätzlich, dass fehlerhafte
oder ungetestete Funktionalität in das endgültige Paket gelangt.

\subsection{Deployment}
\label{sec:Deployment}
Die Bereitstellung der Applikation geschieht in Kollaboration mit dem Team, in welchem das
Projekt durchgeführt wurde. Auf einer Produktivmaschine, die bereits als Webserver für
andere Applikationen fungiert, wird eine weitere Konfiguration angelegt und das Software-Paket der
Applikation installiert. Nach Neustart der zugehörigen Services und Prozesse ist die
Webapplikation intern über den Browser zu erreichen.
\pagebreak
