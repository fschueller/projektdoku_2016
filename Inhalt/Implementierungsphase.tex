% !TEX root = ../Projektdokumentation.tex
\section{Implementierungsphase}
\label{sec:Implementierungsphase}

\subsection{Einleitung}
\label{sec:Einleitung}
Nach der Planungsphase wurde im direkten Anschluss die Implementierungsphase mit folgenden Schritten
eingeleitet:
\begin{enumerate}
	\item Implementierung des Backends (Datenbank, Controllerlogik, Administratorbereich)
	\item Implementierung der Benutzeroberfläche (Bootstrap, Tab-Layout)
	\item Anbindung externer Schnittstellen (Github API (octokit), LDAP-Authentifizierung (devise))
	\item Implementierung der internen Schnittstelle (rabl-gem)
	\item Durchführung einer Testphase
	\item Paketierung der Applikation (Schreiben des Specfiles, Patch)
	\item Deployment auf Produktivmaschine (Konfigurieren des Apache-Webservers, vhost, SSL)
\end{enumerate}

\subsection{Implementierung des Backends}
\label{sec:Implementierung des Backends}
\begin{itemize}
	\item Initiales Rails-Projekt
	\item Einbindung von rspec als Testframework, Schreiben von Modeltests
	\item Initalisierung der Development-Datenbank in SQLite3
	\item Implementierung der Models
	\item Implementierung von Validierungen zur Vermeidung von Falscheingaben (code example)
	\item zugehörige Migrationen
	\item Schreiben der Controllertests
	\item Implementierung der Controllerlogik (code examples)
	\item Implementierung Administratorbereich, Aufteilung des User-Models
\end{itemize}

\subsection{Implementierung der Benutzeroberfläche}
\label{sec:Implementierung der Benutzeroberfläche}
\begin{itemize}
	\item Schreiben von Feature-Tests (capybara) (code example)
	\item Einbindung von Bootstrap
	\item Gestaltung des Layouts als Dashboard, Panel und Tabelemente (LH?) (screenshot)
	\item Nutzung von Partials zur Verringerung der Response-Time
	\item Admin-Tabs nur sichtbar und anzeigbar wenn als Admin authentifiziert (controller-restrained)
\end{itemize}

\subsection{Anbindung externer Schnittstellen}
\label{sec:Anbindung externer Schnittstellen}
\begin{itemize}
	\item Everybody loves tests
	\item LDAP-Konfiguration für Devise zur Mitarbeiterauthentifizierung
	\item first login: Abfrage statischer Informationen wie Name, Firmenemail, Standort; Speichern in DB
	\item bei jedem weiteren Login: dynamische Abfrage des Managers (auch bei Detailansicht in Admin-View)
	\item dynamische Abfrage des Mitarbeiterstatus einmal wöchentlich
	\item Team-view
	\item vgl. entwickelte LDAP-Methode (code example)
	\item Github/Octokit: Abfrage der Organisationszugehörigkeit und Anzeige von mailto an admins
	\item Gleiche Methode für Trello auf Grund von Zeitmangel nicht mehr implementiert
	\item Sidekiq/Sidetiq Jobs (code example)
\end{itemize}

\subsection{Interne Schnittstelle}
\label{sec:Interne Schnittstelle}
\begin{itemize}
	\item Test it!
	\item Formatierung des JSON-Outputs für etsync
	\item Controller response to JSON
	\item rabl-gem zur einfachen Formatierung von JSON mit einfachen Active Record-Queries
	\item Da in Admin-Namespace nur ansprechbar bei Authentifizierung als Administrator
\end{itemize}

\subsection{Testphase}
\label{sec:Testphase}
\begin{itemize}
	\item Gering gehalten, da TDD inkl. Feature-Tests
	\item Rerun der Testsuite
	\item Manueller Testrun mit Usern zum Test der GUI
\end{itemize}

\subsection{Paketierung}
\label{sec:Paketierung}
\begin{itemize}
	\item rpm
	\item specfile (excerpt of specfile)
	\item Automatisierung der Paketierung bei neuem Commit in Repository, nach Run der Tests
	\item Hält Paket auf neuestem Stand
\end{itemize}

\subsection{Deployment}
\label{sec:Deployment}
\begin{itemize}
	\item Deployment in Kollaboration mit SUSE IT-Team, Bereitstellung der Produktivmaschine und
	Produktivdatenbank
	\item Konfigurieren eines Apache-Servers für vhosts
	\item Manuelle Installation der Applikation auf Produktivmaschine
\end{itemize}

% \subsection{Implementierung der Datenstrukturen}
% \label{sec:ImplementierungDatenstrukturen}
%
% \begin{itemize}
% 	\item Beschreibung der angelegten Datenbank (\zB Generierung von \acs{SQL} aus Modellierungswerkzeug oder händisches Anlegen), \acs{XML}-Schemas \usw.
% \end{itemize}
%
%
% \subsection{Implementierung der Benutzeroberfläche}
% \label{sec:ImplementierungBenutzeroberflaeche}
%
% \begin{itemize}
% 	\item Beschreibung der Implementierung der Benutzeroberfläche, falls dies separat zur Implementierung der Geschäftslogik erfolgt (\zB bei \acs{HTML}-Oberflächen und Stylesheets).
% 	\item \Ggfs Beschreibung des Corporate Designs und dessen Umsetzung in der Anwendung.
% 	\item Screenshots der Anwendung
% \end{itemize}
%
% \paragraph{Beispiel}
% Screenshots der Anwendung in der Entwicklungsphase mit Dummy-Daten befinden sich im \Anhang{Screenshots}.
%
%
% \subsection{Implementierung der Geschäftslogik}
% \label{sec:ImplementierungGeschaeftslogik}
%
% \begin{itemize}
% 	\item Beschreibung des Vorgehens bei der Umsetzung/Programmierung der entworfenen Anwendung.
% 	\item \Ggfs interessante Funktionen/Algorithmen im Detail vorstellen, verwendete Entwurfsmuster zeigen.
% 	\item Quelltextbeispiele zeigen.
% 	\item Hinweis: Wie in Kapitel~\ref{sec:Einleitung}: \nameref{sec:Einleitung} zitiert, wird nicht ein lauffähiges Programm bewertet, sondern die Projektdurchführung. Dennoch würde ich immer Quelltextausschnitte zeigen, da sonst Zweifel an der tatsächlichen Leistung des Prüflings aufkommen können.
% \end{itemize}
%
% \paragraph{Beispiel}
% Die Klasse \texttt{Com\-par\-ed\-Na\-tu\-ral\-Mo\-dule\-In\-for\-ma\-tion} findet sich im \Anhang{app:CNMI}.
%
%
% \Zwischenstand{Implementierungsphase}{Implementierung}
