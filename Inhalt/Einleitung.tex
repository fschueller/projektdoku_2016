% !TEX root = ../Projektdokumentation.tex
\section{Einleitung}
\label{sec:Einleitung}

\subsection{Projektumfeld}
\label{sec:Projektumfeld}
Die SUSE Linux GmbH wurde 1992 gegründet und ist seit 2014 eine Tochtergesellschaft der Micro Focus
PLC. Weltweit beschäftigt das Unternehmen etwa 750 Mitarbeiter, welche auf die Standorte Nürnberg,
Prag, Peking und Provo (USA) verteilt sind. Der Hauptsitz der Firma befindet sich in Nürnberg,
ebenso wie der Hauptteil der Softwareentwicklung.

Das Kerngeschäft der SUSE Linux GmbH umfasst die Entwicklung einer Linux-basierten Distribution.
Für Privatkunden werden hierzu die Distributionen der openSUSE-Familie, die größtenteils von der
Community entwickelt werden, bereitgestellt, für Geschäftskunden die Produkte der SUSE Linux
Enterprise-Familie.

Die Entwicklung erfolgt nach dem Open-Source-Prinzip.
Durchgeführt wurde das Projekt im Team SUSE-IT, die neben der Infrastruktur auch die
Entwicklungsumgebung, die sogenannten Engineering Services, bereitstellen.

\subsection{Projektbeschreibung}
\label{sec:Projektbeschreibung}
Zur Entwicklung der Open-Source-Software setzt die SUSE Linux GmbH in Teilen der Entwicklung
auf externe Tools und Services ein. Dies wird besonders im Bereich der Codeentwicklung und -pflege
deutlich, da neben den firmeneigenen Entwicklern auch Partner sowie Open-Source-Community-Mitglieder
mitarbeiten. Hierbei kann es zu der Diskrepanz kommen, dass Entwickler neben ihrem Firmenaccount
auch private Accounts zur Entwicklung benutzen, wodurch eine genaue Zuordnung der Accounts und
ein Überblick über Zugriffsberechtigungen nahezu unmöglich wird.

Zur Lösung dieser Problematik soll eine Applikation entwickelt werden, welche es ermöglicht
eine Zuordnung der firmeninternen Accounts auf mögliche Accounts externer Services durchzuführen.
Das Ergebnis soll mittels einer \acs{API} abrufbar sein.
Dabei soll die Applikation in ihrer Grundfunktion dem Gedanken der \textit{Self-Service-Technologies} folgen,
damit Latenzen bei Änderungen der Accountinformationen minimiert und administratives Personal
entlastet wird. Um dies möglichst plattformunabhängig zu gestalten, erfolgt die Realisierung als
Webapplikation.

Da externe Services eine Zulassung durch den Betriebsrat und die IT-Sicherheit benötigen, beschränkt
sich die erste Version dieser Applikation darauf, die Accountinformationen der Applikationen Github
(verzeichnisorganisierte Versionsverwaltung von Quellcode) und Trello (Tool zur Steuerung von
agilen Software-Projekten) einzubinden. Diese erhielten bereits im Vorfeld dieser Arbeit eine
Zulassung.

\subsection{Ist-Analyse}
\label{sec:Ist-Analyse}
Bei den verwendeten externen Services Github und Trello können Unternehmen zur Abgrenzung und
Verwaltung ihrer firmeninternen Ressourcen und Informationen Organisationen anlegen, zu
denen Accounts von Mitarbeitern hinzugefügt werden und ihnen besondere Zugriffsrechte einzuräumen.
So haben beispielsweise nur Mitarbeiter Schreibrechte auf den Quellcodeverzeichnissen bei Github und
Mitglieder der Open-Source-Community sowie Partner können in der Regel nur Vorschläge, sogenannte
Pull Requests, bei dem jeweiligen Projekt einreichen.

Der Abgleich der internen Mitarbeiter- mit den Userlisten innerhalb der angelegten Firmenorganisationen
erfolgt bisher manuell. Dies bedeutet, dass beispielsweise für jeden neuen
Mitarbeiter persönlich oder schriftlich angefragt werden muss welches User-Alias er bei den
jeweiligen Services führt, um dessen Account dann zur Organisation hinzuzufügen und ihm
Zugriff zu ermöglichen. Ist dieser Aufwand bereits groß, erhöht er sich bei Verlassen eines Mitarbeiters
der Firma, wenn es gegebenenfalls auf Grund fehlender Kontaktmöglichkeit sehr mühsam wird eine Zuordnung
nachzuvollziehen.

Bei Organisationen mit mehreren hundert Mitgliedern entsteht dadurch ein großer zeitlicher und
personeller Aufwand. Durch die ungenaue Beschaffung der benötigten Informationen ist das bisherige
Konzept sehr fehleranfällig und ineffizient durchzuführen.

\subsection{Soll-Konzept}
\label{sec:Soll-Konzept}
Um Personal zu entlasten und den Prozess wesentlich zu vereinfachen, soll eine
zentrale Plattform geschaffen werden, die eine gesammelte Zuordnung der Mitarbeiter zu ihren
externen Nutzerkonten zum Abgleich bereitstellt. Die Datensätze sollen von den jeweiligen
Mitarbeitern selbst verwaltet und gepflegt werden. Dadurch bleibt alleine der Abgleich mit den
Nutzerlisten der bei den externen Services angelegten Organisationen und ein eventuelles Hinzufügen
oder Entfernen des Mitarbeiters aus der Organisation übrig.

Die Plattform soll als Webapplikation gestaltet werden, was eine plattformunabhängige Nutzung
ermöglicht. Zusätzlich soll die Implementierung modular und leicht erweiterbar
sein, um zukünftig geplante Funktionen einfach hinzufügen und warten zu können. Ebenso soll die
Applikation über Tests verfügen, welche jene für andere Entwickler in Zukunft nachvollziehbar macht.

\subsection{Projektschnittstellen}
\label{sec:Projektschnittstellen}
Die Applikation nutzt mehrere APIs um z.B. die Betriebszugehörigkeit oder verschiedene Informationen
aus den Datenbanken der externen Services abzufragen.

Mitarbeiterspezifische Informationen, wie Name, Standort oder Vorgesetzter, werden aus dem
firmeneigenen eDirectory abgefragt.
Zur Abfrage der Organisationszugehörigkeit werden die externen Schnittstellen der genannten Tools
Github und Trello verwendet.

Die Schnittstelle der Applikation, welche die gesammelten Informationen über den Mitarbeiter
bereitstellt, wird in dem Administrationstool etsync konsumiert, welches für die Administratoren
der externen Organisationen bei Github und Trello entwickelt wurde.
