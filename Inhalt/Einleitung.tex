% !TEX root = ../Projektdokumentation.tex
\section{Einleitung}
\label{sec:Einleitung}

\subsection{Projektumfeld}
\label{sec:Projektumfeld}
Die SUSE Linux GmbH wurde 1992 gegründet und ist seit 2014 eine Tochtergesellschaft der Micro Focus
PLC. Weltweit beschäftigt das Unternehmen etwa 750 Mitarbeiter, welche auf die Standorte Nürnberg,
Prag, Peking und Provo (USA) verteilt sind. Der Hauptsitz der Firma befindet sich in Nürnberg,
ebenso wie der Hauptteil der Softwareentwicklung.

Das Kerngeschäft der SUSE Linux GmbH umfasst die Entwicklung einer Linux-basierten Distribution.
Für Privatkunden werden hierzu die Distributionen der openSUSE-Familie, die größtenteils von der
Community entwickelt werden, bereitgestellt. Für Geschäftskunden werden die Produkte der SUSE Linux
Enterprise-Familie gepflegt. Die Softwareentwicklung erfolgt nach dem Open-Source-Prinzip.

Durchgeführt wurde das Projekt im Team SUSE-IT, die neben der Infrastruktur auch die
Entwicklungsumgebung, die sogenannten Engineering Services, bereitstellen.

\subsection{Projektbeschreibung}
\label{sec:Projektbeschreibung}
Zur Entwicklung der Open-Source-Software setzt die SUSE Linux GmbH in Teilen der Entwicklung
auf externe Tools und Services ein. Dies wird besonders im Bereich der Codeentwicklung und -pflege
deutlich, da neben den firmeneigenen Entwicklern auch Partner sowie Open-Source-Community-Mitglieder
mitarbeiten. Hierbei kann es zu der Diskrepanz kommen, dass Entwickler neben ihrem Firmenaccount
auch private Accounts zur Entwicklung benutzen, wodurch eine genaue Zuordnung der Accounts und
ein Überblick über Zugriffsberechtigungen nahezu unmöglich wird.

Zur Lösung dieser Problematik soll eine Applikation entwickelt werden, welche es ermöglicht
eine Zuordnung der firmeninternen Accounts auf mögliche Accounts externer Services durchzuführen.
Das Ergebnis soll mittels einer \acs{API} abrufbar sein.
Dabei soll die Applikation in ihrer Grundfunktion dem Gedanken der \textit{Self-Service-Technologies} folgen,
damit Latenzen bei Änderungen der Accountinformationen minimiert und administratives Personal
entlastet wird. Um dies möglichst plattformunabhängig zu gestalten, erfolgt die Realisierung als
Webapplikation.

Da externe Services eine Zulassung durch den Betriebsrat und IT-Sicherheit und den Datenschutzbeauftragten
benötigen, beschränkt sich die erste Version dieser Applikation darauf, die Accountinformationen der
Applikationen Github (verzeichnisorganisierte Versionsverwaltung von Quellcode) und Trello
(Tool zur Steuerung von agilen Software-Projekten) einzubinden. Diese erhielten bereits im Vorfeld
dieser Arbeit eine Zulassung.

\subsection{Ist-Analyse}
\label{sec:Ist-Analyse}
Bei den genannten externen Services können Unternehmen zur Abgrenzung und
Verwaltung ihrer firmeninternen Entwicklungsprojekte Gruppierungen als sogenannte Organisationen anlegen.
Die externen Accounts von Mitarbeitern können in diese Organisationen aufgenommen werden,
um ihnen damit erweiterte Zugriffsrechte einzuräumen.
So haben beispielsweise nur interne Mitarbeiter Schreibrechte auf den Quellcodeverzeichnissen, während
Mitglieder der Open-Source-Community sowie Partner nur Vorschläge, sogenannte
\textit{Pull Requests}, bei dem jeweiligen Projekt einreichen können.

Der Abgleich der internen Mitarbeiter- mit den Userlisten innerhalb der angelegten Organisationen
erfolgt bisher manuell. Dies bedeutet, dass beispielsweise für jeden neuen
Mitarbeiter persönlich oder schriftlich angefragt werden muss welches User-Alias er bei den
jeweiligen Services führt, um dessen Account dann zur Organisation hinzuzufügen und ihm
Zugriff zu ermöglichen. Ist dieser Aufwand bereits groß, erhöht er sich bei Verlassen eines Mitarbeiters
der Firma, wenn es gegebenenfalls auf Grund fehlender Kontaktmöglichkeit sehr mühsam wird eine Zuordnung
nachzuvollziehen.

Bei Organisationen mit mehreren hundert Mitgliedern entsteht dadurch ein großer zeitlicher und
personeller Aufwand. Durch die dezentrale Beschaffung der benötigten Informationen ist das bisherige
Konzept sehr fehleranfällig und ineffizient durchzuführen.

\subsection{Soll-Konzept}
\label{sec:Soll-Konzept}
Wie in der Ist-Analyse aufgezeigt benötigt der bisherige Prozess viel Zeit und ist kompliziert
durchzuführen. Ziel ist die Vereinfachung des Prozesses und die Entlastung des administrativen Personals.
Die Lösung dafür stellt eine personenbezogene Zuordnung der internen Mitarbeiterdaten zu den jeweiligen
externen Nutzerkonten dar. Gleichzeitig soll durch Anbieten als \textit{Self-Service} eine weitere
Entlastung vorgenommen werden. Hierbei werden Änderungen und Verwaltung der Accountinformationen durch den Mitarbeiter
durchgeführt. Lediglich das Hinzufügen zu oder Entfernen aus den entsprechenden Organisationen obliegt weiterhin den
Administratoren.

Zur Abfrage der mitarbeiterspezifischen Informationen, wie Name, Standort oder Vorgesetzter, kann
das firmeneigene \textit{eDirectory} genutzt werden.

Um dem Mitarbeiter Möglichkeit zur Eigeninitiative einzuräumen, können die Schnittstellen der
genannten externen Tools Github und Trello genutzt werden, um bei Hinzufügen der Nutzerinformation
die Organisationszugehörigkeit abzufragen. Durch eine Abfrage im Hintergrund und Anzeige eines
passenden Hinweises kann der jeweilige Mitarbeiter den Administratoren selbst eine Anfrage schicken.

Um Erreichbarkeit und einfache Bedienbarkeit zu gewährleisten, bietet sich die Umsetzung als
Webapplikation an. Um diese leicht wartbar und für zukünftige Funktionen erweiterbar zu halten, soll
ein modulares Framework genutzt werden.

Die Applikation soll so entwickelt werden, dass sie zukünftig anpassbar und erweiterbar ist. Dies
stellt die zusätzliche Anforderung an den Quellcode, dass dieser in einer entsprechenden nachvollziehbaren
Form geschrieben wird.

Die Schnittstelle der Applikation, welche die gesammelten Informationen über den Mitarbeiter
bereitstellt, wird in dem Administrationstool "etsync" konsumiert, welches für die Administration
externen Tools entwickelt wurde.

\subsection{Projektverantwortlicher}
\label{sec:Projektverantwortlicher}
Durchgeführt wird das Projekt innerhalb der SUSE Linux GmbH im Team SUSE-IT, Engineering Services
Department. Ansprechpartner für das Projekt ist Cornelius Schumacher.

\subsection{Anmerkung zur Dokumentation}
\label{sec:Anmerkung zur Dokumentation}
Einschlägige Fachbegriffe der IT werden ohne Ausschrift im Text verwendet. Zu fachspezifische
Begriffe werden kursiv abgebildet und sind im Glossar erläutert. Eigen- und Modulnamen werden in " "
geschrieben. Begriffe, in diesem Fall besonders aus dem Bereich der verwendeten Programmiersprache,
werden in ihrer englischen Urform gelassen.
\pagebreak
