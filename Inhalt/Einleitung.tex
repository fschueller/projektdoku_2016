% !TEX root = ../Projektdokumentation.tex
\section{Einleitung}
\label{sec:Einleitung}

\subsection{Projektumfeld}
\label{sec:Projektumfeld}
Die SUSE Linux GmbH wurde 1992 gegründet und ist seit 2014 eine Tochtergesellschaft der Micro Focus
PLC. Weltweit beschäftigt das Unternehmen etwa 750 Mitarbeiter, welche auf die Standorte Nürnberg,
Prag, Peking und Provo (USA) verteilt sind. Der Hauptsitz der Firma befindet sich in Nürnberg,
ebenso wie der Hauptteil der Softwareentwicklung.

Das Kerngeschäft der SUSE Linux GmbH umfasst die Entwicklung einer Linux-basierten Distribution.
Für Privatkunden werden hierzu die Distributionen der openSUSE-Familie, die größtenteils von der
Community entwickelt werden, bereitgestellt. Für Geschäftskunden werden die Produkte der SUSE Linux
Enterprise-Familie gepflegt. Die Softwareentwicklung erfolgt nach dem Open-Source-Prinzip.

Für dieses Projekt hervorzuheben ist das Department \anf{Engineering
Services}, welches sich besonders mit dem \anf{SUSE-IT} Team um die
reibungslose Funktion der intern eingesetzten Applikationen kümmert.

\subsection{Projektbeschreibung}
\label{sec:Projektbeschreibung}
Zur Entwicklung der Open-Source-Software setzt die SUSE Linux GmbH in Teilen der Entwicklung
externe Tools und Services ein. Dies wird besonders im Bereich der Codeentwicklung und -pflege
deutlich, da neben den firmeneigenen Entwicklern auch Partner sowie Open-Source-Community-Mitglieder
mitarbeiten. Hierbei kann es zu der Diskrepanz kommen, dass Entwickler neben ihrem Firmenaccount
auch private Accounts zur Entwicklung benutzen. Dies hat zur Folge, dass eine genaue Zuordnung
der Mitarbeiteraccounts zu deren externen Accounts
sowie ein Überblick über die Zugriffsberechtigungen nahezu unmöglich ist.

Zur Lösung dieser Problematik soll eine Applikation entwickelt werden, welche es ermöglicht,
eine Zuordnung der firmeninternen Accounts zu möglichen Accounts externer Services durchzuführen.
Das Ergebnis soll mittels einer Schnittstelle (\acs{API}) abrufbar sein.
Dabei soll die Applikation in ihrer Grundfunktion dem Gedanken der \textit{Self-Service-Technologies} folgen,
damit Latenzen bei Änderungen der Accountinformationen minimiert und administratives Personal
entlastet wird. Um dies möglichst plattformunabhängig zu gestalten, erfolgt die Realisierung als
Webapplikation.

Da externe Services eine Zulassung durch den Betriebsrat, IT-Sicherheit und den Datenschutzbeauftragten
benötigen, beschränkt sich die erste Version dieser Applikation darauf, die Accountinformationen der
Services Github (verzeichnisorganisierte Versionsverwaltung von Quellcode) und Trello
(Tool zur Steuerung von agilen Software-Projekten) einzubinden. Diese erhielten bereits im Vorfeld
dieser Arbeit eine Zulassung.

\subsection{Ist-Analyse}
\label{sec:Ist-Analyse}
Bei den genannten externen Services können Unternehmen zur Abgrenzung und
Verwaltung ihrer firmeninternen Entwicklungsprojekte Gruppierungen, sogenannte Organisationen, anlegen.
Die externen Accounts von Mitarbeitern können in diese Organisationen aufgenommen werden,
um ihnen damit erweiterte Zugriffsrechte einzuräumen.
So haben beispielsweise nur interne Mitarbeiter Schreibrechte auf den Quellcodeverzeichnissen, während
Mitglieder der Open-Source-Community sowie Partner nur Vorschläge, sogenannte
\textit{Pull Requests}, bei dem jeweiligen Projekt einreichen können.

Der Abgleich der internen Mitarbeiter- mit den Userlisten innerhalb der angelegten Organisationen
erfolgt bisher manuell. Das bedeutet dass neue Mitarbeiter persönlich oder virtuell
kontaktiert werden müssen, damit persönliche User-Aliase zur jeweiligen Organisation hinzugefügt
werden können. Hierdurch wird der Zugriff entsprechend der festgelegten Berechtigung ermöglicht.

Dieser Vorgang nimmt mit steigender Nutzer- und Tool-Zahl an Aufwand und Komplexität zu.
Zusätzlich ist durch die dezentrale Beschaffung der benötigten Informationen das bisherige
Konzept sehr fehleranfällig und nur ineffizient durchführbar.

\subsection{Soll-Konzept}
\label{sec:Soll-Konzept}
Ziel des Projektes ist es, eine geeignete Applikation zu entwickeln, die den bisherigen, oben
aufgezeigten Prozess zentralisiert und vereinfacht. Hierdurch wird ebenfalls die Entlastung des
administrativen Personals angestrebt. Die Lösung dafür stellt eine personenbezogene Zuordnung der
internen Mitarbeiterdaten zu den jeweiligen externen Nutzerkonten dar, welche die zusätzlichen folgenden
Anforderungen erfüllen soll:

Die Realisierung erfolgt als \textit{Self-Service}. Hierbei sind Änderungen sowie die Verwaltung
der Accountinformationen durch den Mitarbeiter selbst durchzuführen. Ausnahme stellt das Hinzufügen
zu und Entfernen aus den entsprechenden Organisationen dar.

Zur Abfrage der mitarbeiterspezifischen Informationen, wie Name, Standort oder Vorgesetzter, soll
das firmeneigene \textit{eDirectory} als Datenquelle genutzt werden.

Die Applikation soll zur Funktionsdarstellung die externen Services \anf{Github} und \anf{Trello} einbinden.

Um dem Mitarbeiter Möglichkeit zur Eigeninitiative einzuräumen, können die Schnittstellen der
genannten externen Tools genutzt werden, um bei Hinzufügen der Nutzerinformation
die Organisationszugehörigkeit abzufragen. Durch eine Abfrage im Hintergrund und Anzeige eines
passenden Hinweises kann der jeweilige Mitarbeiter den Administratoren selbst eine Anfrage schicken.

Da die Applikation firmenweit eingesetzt werden soll, muss eine intuitive und lokalisierbare
Bedienbarkeit gewährleistet sein. Die Realisierung als Webapplikation ist daher bevorzugt.
Die eigentliche Lokalisierungsarbeit ist nicht Teil der Projektarbeit.

Die Entwicklung soll auf ein modulares Framework zurückgreifen. Gleichzeitig soll die Applikation
anpassbar und erweiterbar sein. Dies stellt ebenso eine Anforderung an den Quellcode, so dass dieser
in einer entsprechend nachvollziehbaren Form geschrieben sein muss.

Die Informationen, welche die Applikation mittels seiner API bereitstellt, müssen für das Administrationstool
\anf{etsync} konsumierbar und verarbeitbar sein. Dies ist bereits im Unternehmen vorhanden und
ist nicht Teil dieses Projektes.

\subsection{Projektverantwortlicher}
\label{sec:Projektverantwortlicher}
Durchgeführt wird das Projekt innerhalb der SUSE Linux GmbH im Team SUSE-IT, Engineering Services
Department. Ansprechpartner für das Projekt ist Cornelius Schumacher.

\subsection{Anmerkung zur Dokumentation}
\label{sec:Anmerkung zur Dokumentation}
Einschlägige Fachbegriffe der IT werden ohne Ausschrift im Text verwendet. Zu fachspezifische
Begriffe werden kursiv abgebildet und sind im Glossar erläutert. Eigen- und Modulnamen werden in
\anf{ } geschrieben. Begriffe, in diesem Fall besonders aus dem Bereich der verwendeten Programmiersprache,
werden in ihrer englischen Urform gelassen.
\pagebreak
