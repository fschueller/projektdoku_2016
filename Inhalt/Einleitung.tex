% !TEX root = ../Projektdokumentation.tex
\section{Einleitung}
\label{sec:Einleitung}

\subsection{Projektbeschreibung}
\label{sec:Projektbeschreibung}
Zur Entwicklung der Open-Source-Software setzt die SUSE Linux GmbH in Teilen der Entwicklung
auf externe Tools und Services. Dies wird besonders im Bereich der Codeentwicklung und -pflege
deutlich, da neben den firmeneigenen Entwicklern auch Partner sowie Open-Source-Community
mitarbeiten. Hierbei kann es zu der Diskrepanz kommen, dass Entwickler neben ihrem Firmenaccount
auch private Accounts zur Entwicklung benutzen, wodurch eine genaue Zuordnung der Accounts und
ein Überblick über Zugriffsberechtigungen nahezu unmöglich wird.

Zur Lösung dieser Problematik soll eine Applikation entwickelt werden, welche es ermöglicht
eine Zuordnung der firmeninternen Accounts auf mögliche Accounts externer Services durchzuführen.
Das Ergebnis soll mittels einer API abrufbar sein.
Dabei soll die Applikation in ihrer Grundfunktion dem Gedanken der Self-Service-Technologies folgen,
damit Latenzen bei Änderungen der Accountinformationen minimiert und administratives Personal
entlastet wird. Um dies möglichst plattformunabhängig zu gestalten, erfolgt die Realisierung als
Webapplikation.

Da externe Services eine Zulassung durch den Betriebsrat und die IT-Sicherheit benötigen, beschränkt
sich die erste Version darauf, die Accountinformationen der Applikationen Github
(verzeichnisorganisierte Versionsverwaltung von Quellcode) und Trello (Tool zur Steuerung von
agilen Software-Projekten) einzubinden.

\subsection{Projektumfeld}
\label{sec:Projektumfeld}
Die SUSE Linux GmbH wurde 1992 gegründet und ist seit 2014 eine Tochtergesellschaft der Micro Focus
AG. Weltweit beschäftigt das Unternehmen etwa 750 Mitarbeiter, welche auf die Standorte Nürnberg,
Prag, Peking und Provo (USA) verteilt sind. Der Hauptsitz der Firma befindet sich in Nürnberg,
ebenso wie der Hauptteil der Softwareentwicklung.

Das Kerngeschäft der SUSE Linux GmbH umfasst die Entwicklung einer Linux-basierten Distribution.
Für Privatkunden werden hierzu die Distributionen der openSUSE-Familie, die größtenteils von der
Community entwickelt werden, bereitgestellt, für Geschäftskunden die Produkte der SUSE Linux
Enterprise-Familie.

Die Entwicklung erfolgt nach dem Open-Source-Prinzip.
Durchgeführt wurde das Projekt im Team SUSE IT, die neben der Infrastruktur auch die
Entwicklungsumgebung, die sogenannten Engineering Services, bereitstellen.

\subsection{Projektschnittstellen}
\label{sec:Projektschnittstellen}
Die Applikation nutzt mehrere APIs um z.B. die Betriebszugehörigkeit oder verschiedene Informationen
aus den Datenbanken der externen Services abzufragen.
% \begin{itemize}
% 	\item Mit welchen anderen Systemen interagiert die Anwendung (technische Schnittstellen)?
% 	\item Wer genehmigt das Projekt \bzw stellt Mittel zur Verfügung?
% 	\item Wer sind die Benutzer der Anwendung?
% 	\item Wem muss das Ergebnis präsentiert werden?
% \end{itemize}

\subsection{Ist-Analyse}
\label{sec:Ist-Analyse}

\subsection{Soll-Konzept}
\label{sec:Soll-Konzept}
